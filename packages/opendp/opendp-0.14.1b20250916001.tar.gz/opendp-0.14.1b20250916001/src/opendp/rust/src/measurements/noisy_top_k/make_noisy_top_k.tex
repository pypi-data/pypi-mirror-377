\documentclass{article}
% common styling and macros shared by all proof files

\usepackage[top=1in, right=1in, left=1in, bottom=1.5in]{geometry}

\usepackage{amsmath,amsthm,amsfonts,amssymb,amscd}
\usepackage{listings}
\usepackage{hyperref}
\usepackage{xcolor}
\usepackage{xr}

\usepackage{enumerate} 
\usepackage{physics}
\usepackage{fancyhdr}
\usepackage{hyperref}
\usepackage{graphicx}
\usepackage{tcolorbox}
\usepackage{catchfile}
\usepackage{pdftexcmds}
\usepackage[T1]{fontenc}

% hyperref
\hypersetup{
  colorlinks=true,
  linkcolor=blue,
  linkbordercolor={0 0 1}
}

% \contrib macro to indicate inclusion in "contrib".
\usepackage{tcolorbox}
\newtcolorbox{warn_box}{colback=red!5!white,colframe=red!75!black}
\newcommand{\contrib}{{\begin{warn_box}This proof resides in \textbf{``contrib''} because it has not completed the vetting process.\end{warn_box}}} 
\newcommand{\floatingPoint}{{\begin{warn_box}This implementation is susceptible to floating-point vulnerabilities.\end{warn_box}}} 

% asOfCommit macro to version a code dependency. Arguments:
%    #1: relative path to file you are dependent on
%    #2: commit hash it was last edited. If outdated, this should be the second hash in the footnoote. Otherwise,
%            git log -n 1 --pretty=format:%h -- path/to/file.rs
\makeatletter
\ifnum\pdf@shellescape=1
   % "private" command that builds a link to a blob
  \newcommand{\linkOpendpBlob}[3]{%
    \href{https://github.com/opendp/opendp/blob/#1/#2#3}{\path{#3} at commit #1}}

  % latex macro expansion has a separate phase for \input evaluation
  %     immediately evaluate a command to write a temp file to ./out containing the current directory
  \immediate\write18{[ ! -f out/cwd.txt ] && (mkdir -p out && git rev-parse --show-prefix | sed "s|_|\@backslashchar\@backslashchar\@backslashchar_|g" > out/cwd.txt)}
  %     ...and then retrieve the current working directory by loading the temp file
  \CatchFileDef\GitWorkingDir{out/cwd.txt}{\endlinechar=-1}

  % command for building the (up to date) or (outdated) status
  \newcommand{\fileStatus}[2]{%
  \setbox0=\hbox{\input|"git --no-pager log -n1 --pretty='\@percentchar H' #1 | grep -E '^#2.*'"\unskip}\ifdim\wd0=0pt
        (outdated\footnote{See new changes with \texttt{git diff #2..\input|"git --no-pager log -n1 --pretty='\@percentchar h' #1" \GitWorkingDir\path{#1}}})\else
        (up to date)\fi
  }

  \newcommand{\asOfCommit}[2]{%
      % permalink the target
      \linkOpendpBlob{#2}{\GitWorkingDir}{#1}
      % conditionally add (outdated) or (up to date) depending on matching commit hash
      \fileStatus{#1}{#2}%
  }
\else
  % simplified command if shell-escape not enabled
  \newcommand{\asOfCommit}[2]{#1 at commit #2 (unknown status\footnote{Shell-escape is not enabled. Enable \texttt{--shell-escape} to check if this proof is up-to-date with the code.})}
\fi
\makeatother

% \vettingPR macro to link a PR. Arguments:
%    #1: PR number
\newcommand{\vettingPR}[1]{\href{https://github.com/opendp/opendp/pull/#1}{Pull Request \##1}}

% for links to rustdoc items in OpenDP. Arguments:
%    #1: path to item, and designation as trait, struct, fn, etc.
%    #2: item name
\makeatletter
\ifnum\pdf@shellescape=1
  % latex macro expansion has a separate phase for \input evaluation
  %     immediately evaluate a command to write a temp file to ./out containing the base path
  \immediate\write18{[ ! -f out/rustdoc.txt ] && mkdir -p out && ([ -z `kpsewhich --var-value OPENDP_RUSTDOC_PORT` ] && echo "https://docs.rs/opendp/`head -n 1 \@backslashchar`git rev-parse --show-toplevel\@backslashchar`/VERSION | sed 's|.*-dev.*|latest|g'`" || echo "http://localhost:`kpsewhich --var-value OPENDP_RUSTDOC_PORT`") > out/rustdoc.txt}
  %     ...and then retrieve the base path by loading the temp file
  \CatchFileDef\OpenDPRustdocBase{out/rustdoc.txt}{\endlinechar=-1}
\else
  % if shell commands are not enabled, just claim latest
  \newcommand{\OpenDPRustdocBase}{https://docs.rs/opendp/latest}
\fi
\makeatother
\newcommand{\rustdoc}[2]{\href{\OpenDPRustdocBase/opendp/#1.#2.html}{\texttt{#2}}}

% for links to external dependencies. Arguments:
%    #1: crate name
%    #2: path to item, and designation as trait, struct, fn, etc.
%    #3: item name
\newcommand{\docsrs}[3]{\href{https://docs.rs/#1/latest/#1/#2.#3.html}{\texttt{#3}}}

% minted (pseudocode)
\definecolor{codegreen}{rgb}{0,0.6,0}
\definecolor{codegray}{rgb}{0.5,0.5,0.5}
\definecolor{codepurple}{rgb}{0.58,0,0.82}
\definecolor{backcolour}{rgb}{0.95,0.95,0.92}

\lstdefinestyle{mystyle}{
    backgroundcolor=\color{backcolour},   
    commentstyle=\color{codegreen},
    keywordstyle=\color{magenta},
    numberstyle=\tiny\color{codegray},
    stringstyle=\color{codepurple},
    basicstyle=\ttfamily\footnotesize,
    breakatwhitespace=false,         
    breaklines=true,                 
    captionpos=b,                    
    keepspaces=true,                 
    numbers=left,                    
    numbersep=5pt,                  
    showspaces=false,                
    showstringspaces=false,
    showtabs=false,                  
    tabsize=2
}

\lstset{style=mystyle}

% common commands
\theoremstyle{definition}
\newtheorem{theorem}{Theorem}[section]
\newtheorem{lemma}[theorem]{Lemma}
\newtheorem{definition}[theorem]{Definition}
\newtheorem{warning}{Warning}
\newtheorem{corollary}{Corollary}
\newtheorem{proposition}{Proposition}
\newtheorem{remark}{Remark}
\newtheorem{observation}{Observation}
\newtheorem{note}{Note}

\newcommand{\vicki}[1]{{ {\color{olive}{(vicki)~#1}}}}
\newcommand{\hanwen}[1]{{ {\color{purple}{(hanwen)~#1}}}}
\newcommand{\zach}[1]{{ {\color{red}{(zach)~#1}}}}

\newcommand{\MultiSet}{\mathrm{MultiSet}}
\newcommand{\len}{\mathrm{len}}
\newcommand{\din}{\texttt{d\_in}}
\newcommand{\dout}{\texttt{d\_out}}
\newcommand{\T}{\texttt{T} }
\newcommand{\F}{\texttt{F} }
\newcommand{\Map}{\texttt{Map}}
\newcommand{\X}{\mathcal{X}}
\newcommand{\Y}{\mathcal{Y}}
\newcommand{\True}{\texttt{True}}
\newcommand{\False}{\texttt{False}}
\newcommand{\clamp}{\texttt{clamp}}
\newcommand{\function}{\texttt{function}}
\newcommand{\float}{\texttt{float }}
\newcommand{\questionc}[1]{\textcolor{red}{\textbf{Question:} #1}}


\newcommand{\validTransformation}[2]{%
  \begin{theorem}
  For every setting of the input parameters #1 to #2 such that the given preconditions
  hold, #2 raises an error (at compile time or run time) or returns a valid transformation. A valid transformation has the following properties:
  \begin{enumerate}
      \item \textup{(Data-independent runtime errors).}
      For every pair of members $x$ and $x'$ in \texttt{input\_domain},
      $\texttt{invoke}(x)$ and $\texttt{invoke}(x')$ either both return the same error or neither return an error. 

      \item \textup{(Appropriate output domain).} 
      For every member $x$ in \texttt{input\_domain}, $\function(x)$ is in \texttt{output\_domain} or raises a data-independent runtime error.
      
      \item \textup{(Stability guarantee).} 
      For every pair of members $x$ and $x'$ in \texttt{input\_domain} and for every pair $(\din, \dout)$, 
      where \din\ has the associated type for \texttt{input\_metric} and \dout\ has the associated type for \\ \texttt{output\_metric}, 
      if $x, x'$ are \din-close under \texttt{input\_metric}, $\texttt{stability\_map}(\din)$ does not raise an error,
      and $\texttt{stability\_map}(\din) = \dout$, 
      then $\function(x), \function(x')$ are $\dout$-close under \texttt{output\_metric}.
  \end{enumerate}
  \end{theorem}
}


\newcommand{\validMeasurement}[2]{%
  For every setting of the input parameters #1 to #2 such that the given preconditions
  hold, #2 raises an error (at compile time or run time) or returns a valid measurement. A valid measurement has the following properties:
  \begin{enumerate}
      \item \textup{(Data-independent runtime errors).}
      For every pair of members $x$ and $x'$ in \texttt{input\_domain},
      $\texttt{invoke}(x)$ and $\texttt{invoke}(x')$ either both return the same error or neither return an error. 

      \item \textup{(Privacy guarantee).}
      For every pair of members $x$ and $x'$ in \texttt{input\_domain} and for every pair $(\din, \dout)$,
      where \din\ has the associated type for \texttt{input\_metric} and \dout\ has the associated type for \\ \texttt{output\_measure},
      if $x, x'$ are \din-close under \texttt{input\_metric}, $\texttt{privacy\_map}(\din)$ does not raise an error,
      and $\texttt{privacy\_map}(\din) = \dout$,
      then $\function(x), \function(x')$ are $\dout$-close under \texttt{output\_measure}.
  \end{enumerate}
}

\newcommand{\validPostprocessor}[2]{%
  For every setting of the input parameters #1 to #2 such that the given preconditions
  hold, #2 raises an error (at compile time or run time) or returns a valid postprocessor. A valid postprocessor has the following property:
  \begin{enumerate}
      \item \textup{(Data-independent errors).}
      For every pair of members $x$ and $x'$ in \texttt{input\_domain},
      $\function(x), \function(x')$ either both raise the same error, 
      or neither raise an error.
  \end{enumerate}
}

\newcommand{\validOdometer}[2]{%
  For every setting of the input parameters #1 to #2 such that the given preconditions
  hold, #2 raises an error (at compile time or run time) or returns a valid odometer. 
  A valid odometer has the following properties:
  \begin{enumerate}
      \item \textup{(Data-independent runtime errors).}
      For every pair of members $x$ and $x'$ in \texttt{input\_domain},
      $\texttt{invoke}(x)$ and $\texttt{invoke}(x')$ either both return the same error or neither return an error. 

      \item \textup{(Valid odometer queryable).}
      For every member $x$ in \texttt{input\_domain},
      where $\function(x)$ does not raise an error,
      $\function(x)$ returns a valid odometer queryable.
  \end{enumerate}
}

\newcommand{\validOdometerQueryable}[2]{%
  For every setting of the input parameters #1 to #2 such that the given preconditions
  hold, #2 raises a data-independent error or returns a valid odometer queryable (\texttt{queryable}). 
  A valid odometer queryable is an \rustdoc{core/type}{OdometerQueryable} with the following properties:
  \begin{enumerate}
    \item \textup{(Data-independent errors).}
    For every pair of members $x$ and $x'$ in \texttt{input\_domain}, and every query $q$,
    $\texttt{queryable.invoke}_x(q)$ and $\texttt{queryable.invoke}_{x'}(q)$ either both return the same error or neither return an error. 

    \item \textup{(Privacy Guarantee).}
    For every pair of members $x$ and $x'$ in \texttt{input\_domain}, 
    every adversary $\mathcal{A}$, and for every pair $(\din, \dout)$, 
    where \din\ has the associated type for \texttt{input\_metric} and \dout\ has the associated type for \texttt{output\_measure}, 
    if $x$ and $x'$ are \din-close under \texttt{input\_metric}, $\texttt{queryable.eval(privacy\_loss)}$ does not return an error, 
    and $\texttt{queryable.eval(privacy\_loss)} = \dout$, 
    then $\texttt{View}(\mathcal{A} \leftrightarrow \texttt{queryable}_x), \texttt{View}(\mathcal{A} \leftrightarrow \texttt{queryable}_{x'})$ are $\dout$-close under \texttt{output\_measure},
    where $\texttt{queryable}_x$ denotes all messages received by $\mathcal{A}$ under $x$.
  \end{enumerate}
}

\allowdisplaybreaks

\title{\texttt{fn make\_noisy\_top\_k}}
\author{Michael Shoemate}
\begin{document}
\maketitle

Proves soundness of \rustdoc{measurements/fn}{make\_noisy\_top\_k}
in \asOfCommit{mod.rs}{f5bb719}.\\
\texttt{make\_noisy\_top\_k} returns a Measurement that
noisily selects the indices of the greatest scores from a vector of input scores.

\section{Background}
This mechanism fulfills the same purpose as the exponential mechanism,
where the release is the best candidate's index from a finite set.
The naive implementation of the exponential mechanism samples an index $k$ from $[m] = {1, \ldots, m}$,
where $m$ is the number of candidates,
with probability $p_i$ assigned to each candidate's index $i$ as a function of their score $s_i$.
The output is drawn via inverse transform sampling
by outputting the smallest index $k$ for which the cumulative probability is greater than some $u \sim \mathrm{Uniform}(0, 1)$.
\begin{equation}
    \label{m-naive}
    \mathcal{M}_{\mathrm{naive}}([s_1, \ldots, s_m]) = \min \{k: \sum_{i=1}^k p_i \ge u \}
\end{equation}

The probability of index $k$ being selected is the normalization of its likelihood $\exp(s_k / \tau)$.
As a candidate's score $s_k$ increases, the candidate becomes exponentially more likely to be selected:
\begin{equation}
    \label{prob-of-k}
    p_k = \frac{\exp(s_k / \tau)}{\sum_{i = 1}^m \exp(s_i / \tau)}
\end{equation}

This equation introduces a new temperature parameter, $\tau$,
which calibrates how distinguishable scores are from each other.
As temperature increases, the categorical output distribution tends towards higher entropy/uniformity and becomes more privacy-preserving.
As temperature decreases, the categorical distribution tends towards a one-hot vector
(where each candidate has zero probability, except for the candidate with the maximum score, which has probability one),
becoming less private.
Temperature is related to the privacy loss parameter (\dout) and sensitivity of the scoring function ($\Delta$) as follows:

\begin{equation}
    \tau = \Delta / \dout
\end{equation}

A precise definition of $\Delta$ will come later and is captured by the metrics we use on the score vector $s = [s_1, \ldots, s_m]$.
When $\dout$ increases, temperature decreases, and candidates become more distinguishable from each other.
We also divide scores by their global sensitivity to normalize the sensitivity to one.
In the differential privacy literature for the exponential mechanism, the sensitivity is often multiplied by two.
In OpenDP's \texttt{make\_noisy\_top\_k} this factor is bundled into the $\Delta$ term, which is expressed in terms of a metric that captures monotonicity.

\subsection{Sampling Vulnerabilities}

In practice, computing $\exp(s_i / \tau)$ is prone to
zero underflow (where a non-zero quantity rounds down to zero)
and overflow (where a large finite quantity is replaced with infinity)
due to finite/limited data representation.
Specifically, a scaled score $s_i / \tau$ of just $-709$ underflows to zero and $+710$ overflows to infinity when stored in a 64-bit float.

A simple improvement is to shift the scores by subtracting the greatest score from all scores.
In idealized arithmetic, the resulting probabilities are not affected by shifts in the underlying scores.
On finite data types, this shift prevents a catastrophic overflow, but makes underflow more likely,
causing tail values of the distribution to round to zero.
The inverse transform sampling step is also subject to accumulated rounding errors from the arithmetic and sum,
which influence the likelihood of being chosen.

These potential vulnerabilities can be addressed via the Gumbel-max trick.
The naive mechanism $\mathcal{M}_\mathrm{naive}$ implemented with infinite-precision arithmetic
is equivalent in distribution to the following mechanism:
\begin{equation}
    \mathcal{M}([s_1, \ldots, s_m]) = \mathrm{argmax}_i g_i,
\end{equation}
where each $g_i \sim \mathrm{Gumbel}(\mu = s_i, \beta = \tau)$.

\subsection{Noise Distribution}
\texttt{make\_noisy\_top\_k} can also be configured to satisfy
either \rustdoc{measures/struct}{MaxDivergence} or \rustdoc{measures/struct}{ZeroConcentratedDivergence}.
Gumbel noise is used when \texttt{output\_measure} is \texttt{ZeroConcentratedDivergence},
and exponential noise is used when \texttt{output\_measure} is \texttt{MaxDivergence}.
These choices of noise distributions minimize the necessary noise variance for their respective privacy measures.

Since the permute-and-flip mechanism is equivalent to report noisy max exponential,
and it can be implemented with discrete distributions,
the permute-and-flip mechanism is used instead.

\subsection{Top K}
In the case of Gumbel noise, the distribution of the top k indices is equivalent to adding gumbel noise once,
and then returning the top k indices.
This one-shot mechanism avoids needing to peel the selected candidate from the candidate set and re-run the mechanism.

However, this peeling routine is still used for the permute-and-flip mechanism,
to release the top k elements.
The privacy argument proceeds via composition.

\section{Hoare Triple}
\subsection*{Precondition}
\subsubsection*{Compiler-verified}
\begin{itemize}
    \item \texttt{MO} is a type with trait \rustdoc{measurements/make\_noisy\_top\_k/trait}{TopKMeasure}
    \item \texttt{TIA} (atomic input type) is a type with trait \rustdoc{traits/trait}{Number}
\end{itemize}

\subsubsection*{Caller-verified}
None

\subsection*{Pseudocode}
\label{sec:python-pseudocode}
\lstinputlisting[language=Python,firstline=2,escapechar=|]{./pseudocode/make_noisy_top_k.py}

\subsection*{Postcondition}

\begin{theorem}
    \label{postcondition}
    \validMeasurement{\texttt{input\_domain, input\_metric, output\_measure, k, scale, negate, MO, TIA}}{\texttt{make\_noisy\_top\_k}}
\end{theorem}

\begin{proof}[Proof of data-independent errors.]
    By the postcondition of \rustdoc{measurements/noisy_top_k/trait}{TopKMeasure}\texttt{noisy\_top\_k},
    the only source of error is due to entropy exhaustion, which could be data-dependent,
    due to differing number of expected random draws depending on the input dataset.

    Therefore, the mechanism only satisfies the requirement for data-independent errors conditioned on entropy not being exhausted.
\end{proof}

\begin{proof}[Proof of privacy guarantee.]
    When \din\ is zero, by line~\ref{din-non-zero}, the privacy loss is zero, satisfying the postcondition.
    Otherwise when scale is zero, by line~\ref{scale-zero}, the privacy loss is infinite, also satisfying the postcondition.

    By the checks on lines~\ref{check-non-nan} and \ref{check-non-negative-scale},
    the preconditions for \rustdoc{measurements/noisy_top_k/trait}{TopKMeasure}\texttt{noisy\_top\_k} are satisfied.
    Additionally by the checks on lines~\ref{din-non-neg}, \ref{din-non-zero} and \ref{scale-zero},
    the preconditions for \rustdoc{measurements/noisy_top_k/trait}{TopKMeasure}\texttt{privacy\_map} are satisfied.

    By the postcondition of \rustdoc{measurements/noisy_top_k/trait}{TopKMeasure} and adaptive composition,
    the \dout on line~\ref{fn-privacy-map-call} satisfies the postcondition.
\end{proof}

\end{document}

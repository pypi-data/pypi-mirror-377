
\documentclass[12pt]{article}
%%%%%%%%%%%%%%%%%%%%%%%%%%%%%%%%%%%%%%%%%%%%%%%%%%%%%%%%%%%%%%%%%%%%%%%%%%%%%%%%%%%%%%%%%%%%%%%%%%%%%%%%%%%%%%%%%%%%%%%%%%%%%%%%%%%%%%%%%%%%%%%%%%%%%%%%%%%%%%%%%%%%%%%%%%%%%%%%%%%%%%%%%%%%%%%%%%%%%%%%%%%%%%%%%%%%%%%%%%%%%%%%%%%%%%%%%%%%%%%%%%%%%%%%%%%%
\usepackage{amsfonts}
\usepackage{amssymb}
\usepackage{sw20elba}

%TCIDATA{OutputFilter=LATEX.DLL}
%TCIDATA{Version=5.50.0.2890}
%TCIDATA{<META NAME="SaveForMode" CONTENT="1">}
%TCIDATA{BibliographyScheme=Manual}
%TCIDATA{Created=Friday, July 19, 2013 11:25:23}
%TCIDATA{LastRevised=Friday, July 19, 2013 12:46:00}
%TCIDATA{<META NAME="GraphicsSave" CONTENT="32">}
%TCIDATA{<META NAME="DocumentShell" CONTENT="Articles\SW\mrvl">}
%TCIDATA{CSTFile=LaTeX article (bright).cst}

\newtheorem{theorem}{Theorem}
\newtheorem{axiom}[theorem]{Axiom}
\newtheorem{claim}[theorem]{Claim}
\newtheorem{conjecture}[theorem]{Conjecture}
\newtheorem{corollary}[theorem]{Corollary}
\newtheorem{definition}[theorem]{Definition}
\newtheorem{example}[theorem]{Example}
\newtheorem{exercise}[theorem]{Exercise}
\newtheorem{lemma}[theorem]{Lemma}
\newtheorem{notation}[theorem]{Notation}
\newtheorem{problem}[theorem]{Problem}
\newtheorem{proposition}[theorem]{Proposition}
\newtheorem{remark}[theorem]{Remark}
\newtheorem{solution}[theorem]{Solution}
\newtheorem{summary}[theorem]{Summary}
\newenvironment{proof}[1][Proof]{\noindent\textbf{#1.} }{{\hfill $\Box$ \\}}
\input{tcilatex}
\addtolength{\textheight}{30pt}

\begin{document}

\title{Algebra 6.114}
\author{Michael Vaughan-Lee}
\date{July 2013}
\maketitle

Algebra 6.114 has presentation%
\[
\langle a,b,c\,|\,pa-ba,\,pb-cb,\,pc-kba-ca,\,\text{class }2\rangle
\;(k=0,1,\ldots ,p-1). 
\]

Over all $p$ values of $k$, algebra 6.114 has $4p-4$ descendants of order $%
p^{7}$ and $p$-class 3. The cases $k=-1$ and $k=3$ are straightforward, but
things are more complicated when $k\neq -1,3$. In these cases we have a
parametrized family of algebras%
\[
\langle a,b,c\,|\,bac-zbab,pa-ba,\,pb-cb,\,pc-kba-ca,\,\text{class }3\rangle
,
\]%
where (for a given $k\neq -1,3$) $z$ and $z^{\prime }$ define isomorphic
algebras if the ratios $1:z$ and $1:z^{\prime }$ are in the same orbit of
ratios $\alpha :\beta $ under the action%
\[
\left( 
\begin{array}{c}
\alpha  \\ 
\beta 
\end{array}%
\right) \rightarrow A\left( 
\begin{array}{c}
\alpha  \\ 
\beta 
\end{array}%
\right) 
\]%
where $A$ equals%
\[
\left( 
\begin{array}{cc}
k-1 & 1 \\ 
-1 & 0%
\end{array}%
\right) \text{ or }\left( 
\begin{array}{cc}
k^{2}-2k & k-1 \\ 
1-k & -1%
\end{array}%
\right) \text{ or }\left( 
\begin{array}{cc}
\allowbreak \left( 1+\gamma k\right) \left( \gamma k-2\gamma +1\right)  & 
\allowbreak \gamma \left( \gamma k+2-\gamma \right)  \\ 
-\gamma \left( \gamma k+2-\gamma \right)  & -\left( -1+\gamma \right) \left(
\gamma +1\right) 
\end{array}%
\right) 
\]%
with $\gamma \neq -1$ and $\gamma $ not a root of $\allowbreak \gamma
^{2}+(k-1)\gamma +1=0$. (Note that the ratio $1:0$ is in the same orbit as
the ratio $0:1$.)

A \textsc{Magma} program to compute a set of representative pairs $(k,z)$ is
given in Notes6.114.m.

\end{document}
